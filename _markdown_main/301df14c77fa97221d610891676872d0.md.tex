%%begin novalidate\markdownRendererInterblockSeparator
{}\markdownRendererHeadingThree{We can do that?}\markdownRendererInterblockSeparator
{}\markdownRendererUlBegin
\markdownRendererUlItem Available themes\markdownRendererUlItemEnd 
\markdownRendererUlItem [x] Catppuccin-Mocha\markdownRendererUlItemEnd 
\markdownRendererUlItem [x] Catppuccin-Latte\markdownRendererUlItemEnd 
\markdownRendererUlItem [x] Decay-Green\markdownRendererUlItemEnd 
\markdownRendererUlItem [x] Rosé-Pine\markdownRendererUlItemEnd 
\markdownRendererUlItem [x] Tokyo-Night\markdownRendererUlItemEnd 
\markdownRendererUlItem [x] Material-Sakura\markdownRendererUlItemEnd 
\markdownRendererUlItem [x] Graphite-Mono\markdownRendererUlItemEnd 
\markdownRendererUlItem [x] Cyberpunk-Edge\markdownRendererUlItemEnd 
\markdownRendererUlItem [] Nordic-Blue (maybe later)\markdownRendererUlItemEnd 
\markdownRendererUlItem Yeah to some extent, with \texttt{markdown} package :-)\markdownRendererUlItemEnd 
\markdownRendererUlItem \markdownRendererStrongEmphasis{$\hash$} and \markdownRendererStrongEmphasis{$\hash\hash$} for section and subsection headers (in ToC)\markdownRendererUlItemEnd 
\markdownRendererUlItem Redefine \markdownRendererStrongEmphasis{$\hash\hash\hash$} to start a frame and frametitle\markdownRendererUlItemEnd 
\markdownRendererUlItem (Nested) bullet and numbered lists\markdownRendererUlItemEnd 
\markdownRendererUlItem Text formatting (\markdownRendererEmphasis{italic}, \markdownRendererStrongEmphasis{bold becomes italic + alerted})\markdownRendererUlItemEnd 
\markdownRendererUlItem Redefine \markdownRendererStrongEmphasis{$\hash\hash\hash\hash$} to start a block with title \linebreak and \markdownRendererStrongEmphasis{\texttt{-{}-{}-{}-}} to end the block\markdownRendererUlItemEnd 
\markdownRendererUlItem \markdownRendererStrongEmphasis{\markdownRendererEmphasis{Compile with \texttt{-{}-shell-escape}}} (Overleaf does this already)\markdownRendererUlItemEnd 
\markdownRendererUlItem (Alternative approaches: Pandoc, wikitobeamer)\markdownRendererUlItemEnd 
\markdownRendererUlEnd \markdownRendererInterblockSeparator
{}\end{frame}\markdownRendererInterblockSeparator
{}%%%%%%%%%%%%%%%%%%%%%%\markdownRendererInterblockSeparator
{}\markdownRendererHeadingThree{Caveats}\markdownRendererInterblockSeparator
{}\markdownRendererUlBegin
\markdownRendererUlItem Nothing too complicated!\markdownRendererUlItemEnd 
\markdownRendererUlItem No verbatim or fragile stuff!\markdownRendererUlItemEnd 
\markdownRendererUlItem No $\hash$ and \textunderscore{} characters!\linebreak (I used \markdownRendererCodeSpan{\markdownRendererDollarSign{}\markdownRendererBackslash{}hash\markdownRendererDollarSign{}} and \markdownRendererCodeSpan{\markdownRendererBackslash{}textunderscore})\markdownRendererUlItemEnd 
\markdownRendererUlItem Can't pass options to frames\markdownRendererUlItemEnd 
\markdownRendererUlItem \markdownRendererStrongEmphasis{Need to write \texttt{\textbackslash end\string{frame\string}} manually!}\markdownRendererUlItemEnd 
\markdownRendererUlEnd \markdownRendererInterblockSeparator
{}\end{frame}\markdownRendererInterblockSeparator
{}%%%%%%%%%%%%%%%%%%%%%%\markdownRendererInterblockSeparator
{}%%% # and ## can still be used as sections and subsections if you prefer\markdownRendererInterblockSeparator
{}\markdownRendererHeadingOne{Example}\markdownRendererInterblockSeparator
{}\markdownRendererHeadingTwo{Proposed Menus}\markdownRendererInterblockSeparator
{}%%% ### starts a frame + frametitle\markdownRendererInterblockSeparator
{}\markdownRendererHeadingThree{Breakfast Menu}\markdownRendererInterblockSeparator
{}%%% bulleted lists as usual\markdownRendererInterblockSeparator
{}\markdownRendererUlBegin
\markdownRendererUlItem Eggs\markdownRendererUlItemEnd 
\markdownRendererUlItem scrambled\markdownRendererUlItemEnd 
\markdownRendererUlItem sunny-side-up\markdownRendererUlItemEnd 
\markdownRendererUlItem Coffee\markdownRendererUlItemEnd 
\markdownRendererUlItem Americano\markdownRendererUlItemEnd 
\markdownRendererUlItem Long black\markdownRendererUlItemEnd 
\markdownRendererUlItem Tea\markdownRendererUlItemEnd 
\markdownRendererUlItem Darjeeling\markdownRendererUlItemEnd 
\markdownRendererUlItem English Breakfast\markdownRendererUlItemEnd 
\markdownRendererUlEnd \markdownRendererInterblockSeparator
{}%%% Due to the complicatedness of beamer frames, \end{frame} MUST appear in the source code itself and cannot be "hidden" in another command\markdownRendererInterblockSeparator
{}\end{frame}\markdownRendererInterblockSeparator
{}%%%%%%%%%%%\markdownRendererInterblockSeparator
{}\markdownRendererHeadingThree{Lunch Menu}\markdownRendererInterblockSeparator
{}\markdownRendererUlBegin
\markdownRendererUlItem Spaghetti\markdownRendererUlItemEnd 
\markdownRendererUlItem Bolognese\markdownRendererUlItemEnd 
\markdownRendererUlItem Aglio olio\markdownRendererUlItemEnd 
\markdownRendererUlItem Sandwiches\markdownRendererUlItemEnd 
\markdownRendererUlItem Egg\markdownRendererUlItemEnd 
\markdownRendererUlItem Ham\markdownRendererUlItemEnd 
\markdownRendererUlItem Tuna\markdownRendererUlItemEnd 
\markdownRendererUlEnd \markdownRendererInterblockSeparator
{}\end{frame}\markdownRendererInterblockSeparator
{}%%%%%%%%%%%\markdownRendererInterblockSeparator
{}\markdownRendererHeadingTwo{Budgeting}\markdownRendererInterblockSeparator
{}\markdownRendererHeadingThree{Projected Profit}\markdownRendererInterblockSeparator
{}\markdownRendererOlBegin
\markdownRendererOlItemWithNumber{1}And the answer is\markdownRendererEllipsis{}\markdownRendererOlItemEnd 
\markdownRendererOlItemWithNumber{2}$f(x)=\sum_{n=0}^\infty\frac{f^{(n)}(a)}{n!}(x-a)^n$\markdownRendererOlItemEnd 
\markdownRendererOlItemWithNumber{3}How do we \markdownRendererEmphasis{know} that?\markdownRendererOlItemEnd 
\markdownRendererOlItemWithNumber{4}\markdownRendererStrongEmphasis{Maths!}\markdownRendererOlItemEnd 
\markdownRendererOlEnd \markdownRendererInterblockSeparator
{}\end{frame}\markdownRendererInterblockSeparator
{}\markdownRendererHeadingThree{Testing blocks}\markdownRendererInterblockSeparator
{}\markdownRendererHeadingFour{This is a block!}\markdownRendererInterblockSeparator
{}\markdownRendererUlBegin
\markdownRendererUlItem Here is some content.\markdownRendererUlItemEnd 
\markdownRendererUlItem Here's more contents.\markdownRendererUlItemEnd 
\markdownRendererUlEnd \markdownRendererInterblockSeparator
{}\markdownRendererHorizontalRule{}\markdownRendererInterblockSeparator
{}\end{frame}\markdownRendererInterblockSeparator
{}\markdownRendererHeadingThree{Citations}\markdownRendererInterblockSeparator
{}\markdownRendererUlBegin
\markdownRendererUlItem This is a citation \markdownRendererCite{1}+{}{}{novotny:2017}\markdownRendererUlItemEnd 
\markdownRendererUlItem Works like \markdownRendererCodeSpan{natbib} syntax \markdownRendererCite{1}+{see}{p.26}{novotny:2019}\markdownRendererUlItemEnd 
\markdownRendererUlEnd \markdownRendererInterblockSeparator
{}\end{frame}\markdownRendererInterblockSeparator
{}\markdownRendererHeadingThree{Pipe Tables}\markdownRendererInterblockSeparator
{}\markdownRendererUlBegin
\markdownRendererUlItem Use \markdownRendererCodeSpan{pipeTables} and \markdownRendererCodeSpan{tableCaptions} options\markdownRendererUlItemEnd 
\markdownRendererUlItem Available since \markdownRendererCodeSpan{markdown} v2.8.0\markdownRendererUlItemEnd 
\markdownRendererUlEnd \markdownRendererInterblockSeparator
{}\markdownRendererTable{Demonstration of pipe table syntax.}{4}{4}{rldc}%
{{Right}%
{Left}%
{Default}%
{Center}%
}%
{{12}%
{12}%
{12}%
{12}%
}%
{{123}%
{123}%
{123}%
{123}%
}%
{{1}%
{1}%
{1}%
{1}%
}%
\markdownRendererInterblockSeparator
{}\end{frame}\markdownRendererInterblockSeparator
{}%%novalidate\relax