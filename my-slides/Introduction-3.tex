\documentclass[\string~/GitHub/sthlmNordBeamerTheme/sthlmNordLightDemo.tex]{subfiles}

\begin{document}
%=-=-=-=-=-=-=-=-=-=-=-=-=-=-=-=-=-=-=-=-=-=-=-=-=-=-=-=-=-=-=-=-=-=-=-=-=-=-=-=
%   FRAME START   -=-=-=-=-=-=-=-=-=-=-=-=-=-=-=-=-=-=-=-=-=-=-=-=-=-=-=-=-=-=-=
\begin{frame}[c]{Composition Packages}

    \begin{itemize}
		
            \item The first component is Virtual Flow Simulator (VFS)\footnote{\url{https://www.osti.gov/biblio/1312901}}, is a computational fluidd mechanics (CFD) package that is based on the Curvilinear Immersed Boundary (CURVIB) method to handle geometrically complex and moving domains. 
            
            \item The module that we are working with is a module of VFS that has the capability of fluid-structure interaction (FSI) of solid and deformable bodies. This module has been forked and personally maintained by Dr. Trung B. Le.
		
            \item Before building FSI, a set of two prerequisites are PETSC \footnote{\url{https://petsc.org/release/}} (a suite of library for the scalable solution of scientific applications modeled by partial differential equations) version 3.1 patch 8, and Eigen\footnote{\url{https://gitlab.com/libeigen/eigen}} (a C++ template library for linear algebra: matrices, vectors, numerical solvers, etc.)
            
    \end{itemize}
 
\end{frame}
%   FRAME END   --==-=-=-=-=-=-=-=-=-=-=-=-=-=-=-=-=-=-=-=-=-=-=-=-=-=-=-=-=-=-=
%=-=-=-=-=-=-=-=-=-=-=-=-=-=-=-=-=-=-=-=-=-=-=-=-=-=-=-=-=-=-=-=-=-=-=-=-=-=-=-=
\end{document}