\documentclass[\string~/GitHub/sthlmNordBeamerTheme/sthlmNordLightDemo.tex]{subfiles}

\begin{document}
%   FRAME START   -=-=-=-=-=-=-=-=-=-=-=-=-=-=-=-=-=-=-=-=-=-=-=-=-=-=-=-=-=-=-=
\begin{frame}[c]{Approaches (cont.)}
    
    \begin{block}{Approach 2}
        In this approach, AMR algorithms are implemented into JESSFIF.jl program.
    \end{block}
    
    \begin{itemize}
        \item Pros:
            \begin{itemize}
                \item Do not require the understanding of the whole AMReX library of classes in order to implement AMR.
                \item Developers have less rigid data structure.
                \item The code could be easier to develop or maintain in the without dependence on AMReX.
            \end{itemize}
        \item Cons:
            \begin{itemize}
                \item Time constraint since:
                    \begin{itemize}
                        \item LOTS of coding!!!
                        \item LOTS of documentation!!!
                    \end{itemize}
                \item Julia Lang might not have the proficient libraries for certain features.
            \end{itemize}
    \end{itemize}
 
\end{frame}
%   FRAME END   --==-=-=-=-=-=-=-=-=-=-=-=-=-=-=-=-=-=-=-=-=-=-=-=-=-=-=-=-=-=-=
\end{document}