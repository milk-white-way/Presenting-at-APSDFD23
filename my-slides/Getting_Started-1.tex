\documentclass[\string~/GitHub/sthlmNordBeamerTheme/sthlmNordLightDemo.tex]{subfiles}

\begin{document}
%   FRAME START   -=-=-=-=-=-=-=-=-=-=-=-=-=-=-=-=-=-=-=-=-=-=-=-=-=-=-=-=-=-=-=
\begin{frame}[c]{Approaches}
    
    \begin{block}{Approach 1}
        In this approach, several modules and algorithms from the FSI code are merged into AMReX library.
    \end{block}
    
    \begin{itemize}
        \item Pros:
            \begin{itemize}
                \item AMR capability is already implemented, no need to code from scratch.
                \item Data parallelism can be automatically handled by AMReX classes.
                \item AMReX has better documentation than FSI code.
            \end{itemize}
        \item Cons:
            \begin{itemize}
                \item FSI algorithms in AMReX works differently from FSI code. Being able to implement the Curvilinear Coordinate system in AMReX is the key to deploy Sharp Curvilinear Immersed Boundary method.
                \item Aligning data structures between AMReX and FSI could be a big challenge due to incompatibilities.
                \item Adding features to AMReX's documentation.
            \end{itemize}
    \end{itemize}
 
\end{frame}
%   FRAME END   --==-=-=-=-=-=-=-=-=-=-=-=-=-=-=-=-=-=-=-=-=-=-=-=-=-=-=-=-=-=-=
\end{document}