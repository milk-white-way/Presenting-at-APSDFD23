\documentclass[\string~/GitHub/sthlmNordBeamerTheme/sthlmNordLightDemo.tex]{subfiles}

\begin{document}
%   FRAME START   -=-=-=-=-=-=-=-=-=-=-=-=-=-=-=-=-=-=-=-=-=-=-=-=-=-=-=-=-=-=-=
\begin{frame}[allowframebreaks, allowdisplaybreaks, t, fragile]{Julian Solver}{Packages}

	\begin{center}
        \begin{minted}[linenos, breaklines, fontfamily=helvetica, fontsize=\scriptsize]{julia}
            ## Turn these lines on at the first runtime:
            #using Pkg
            #Pkg.add("LinearAgebra")
            #Pkg.add("SparseArrays")
            #Pkg.add("Printf")
            #Pkg.add("TickTock")
            #Pkg.add("DelimitedFiles")
            #Pkg.add("Debugger")
            #Pkg.add("Makie")
            
            using LinearAlgebra
            using SparseArrays
            using Printf
            using TickTock
            using DelimitedFiles
            using Debugger
            using Makie
        \end{minted}
    \end{center}

\end{frame}
%   FRAME END   --==-=-=-=-=-=-=-=-=-=-=-=-=-=-=-=-=-=-=-=-=-=-=-=-=-=-=-=-=-=-=

%   FRAME START   -=-=-=-=-=-=-=-=-=-=-=-=-=-=-=-=-=-=-=-=-=-=-=-=-=-=-=-=-=-=-=
\begin{frame}[allowframebreaks, allowdisplaybreaks, t, fragile]{Julian Solver}{Constants and Variables}
    
    \begin{center}
        \begin{minted}[linenos, breaklines, fontfamily=helvetica, fontsize=\tiny]{julia}
            ## SCHEME CONSTANTS
            const NK_MAX_ITERATION = 100
            const NK_TOLERANCE = 1.0e-5
            const RK_MAX_ITERATION = 20
            const RK_TOLERANCE = 1.0e-7

            ## PROBLEM CONSTANTS
            const SYS2D_LENGTH_X = 1.0
            const SYS2D_LENGTH_Y = 1.0
            const SYS2D_DIM_X = 61 
            const SYS2D_DIM_Y = 61
            const SYS2D_GHOST_X = SYS2D_DIM_X + 2
            const SYS2D_GHOST_Y = SYS2D_DIM_Y + 2
            const SYS2D_STEP_X = SYS2D_LENGTH_X / (SYS2D_DIM_X - 1)
            const SYS2D_STEP_Y = SYS2D_LENGTH_Y / (SYS2D_DIM_Y - 1)
            const STEP_TIME = 0.008
            const ENDO_TIME = 4000 # seconds
            const REN_NUM = 100
            const DYN_VIS = 1
            const CONVECTIVE_COEFFICIENT = 0.128 # 1/8
            const OUTPUT_FREQUENCY = 2000
        \end{minted}
    \end{center}
		
	\framebreak
 
    \begin{center}
        \begin{minted}[linenos, breaklines, fontfamily=helvetica, fontsize=\tiny]{julia}
            ## Creating solution structure:
            mutable struct Solution2D
            length_x::Float64
            length_y::Float64
            m2::UInt64
            n2::UInt64
            dx::Float64
            dy::Float64
            dt::Float64
            et::Float64
            ren::UInt64
            vis::Float64

            ucat_x::Matrix{Float64}
            ucat_y::Matrix{Float64}
            ucur_x::Matrix{Float64}
            ucur_y::Matrix{Float64}
            ubcs_x::Matrix{Float64}
            ubcs_y::Matrix{Float64}
            uint_x::Matrix{Float64}
            uint_y::Matrix{Float64}
            pres::Matrix{Float64}
            du_x::Matrix{Float64}
            du_y::Matrix{Float64}
            end
        \end{minted}
    \end{center}
    
\end{frame}
%   FRAME END   --==-=-=-=-=-=-=-=-=-=-=-=-=-=-=-=-=-=-=-=-=-=-=-=-=-=-=-=-=-=-=

%   FRAME START   -=-=-=-=-=-=-=-=-=-=-=-=-=-=-=-=-=-=-=-=-=-=-=-=-=-=-=-=-=-=-=
\begin{frame}[allowframebreaks, allowdisplaybreaks, t, fragile]{Julian Solver}{Initialization}

    \begin{center}
        \begin{minted}[linenos, breaklines, fontfamily=helvetica, fontsize=\scriptsize]{julia}
            ## Apply zero initialization
            for ii = 1:SYS2D_GHOST_X
                for jj = 1:SYS2D_GHOST_Y
                    Ucat_x[ii, jj] = 0
                    Ucat_y[ii, jj] = 0
                    Ucur_x[ii, jj] = 0
                    Ucur_y[ii, jj] = 0
                    Ubcs_x[ii, jj] = 0
                    Ubcs_y[ii, jj] = 0
                    Uint_x[ii, jj] = 0
                    Uint_y[ii, jj] = 0
                    Pres[ii, jj] = 0
                    dU_x[ii, jj] = 0
                    dU_y[ii, jj] = 0
                end
            end         
        \end{minted}
    \end{center}
    
\end{frame}
%   FRAME END   --==-=-=-=-=-=-=-=-=-=-=-=-=-=-=-=-=-=-=-=-=-=-=-=-=-=-=-=-=-=-=

%   FRAME START   -=-=-=-=-=-=-=-=-=-=-=-=-=-=-=-=-=-=-=-=-=-=-=-=-=-=-=-=-=-=-=
\begin{frame}[allowframebreaks, allowdisplaybreaks, t, fragile]{Julian Solver}{Call Sequence}

    \begin{center}
        \begin{minted}[linenos, breaklines, fontfamily=helvetica, fontsize=\tiny]{julia}
            while ( current_time < CSolution.et )
                @printf("Time step: %d \n", timestep)
                global current_time = timestep*CSolution.dt
                
                let du_x = deepcopy(CSolution.du_x)
                du_y = deepcopy(CSolution.du_y)
                u_pre_x = deepcopy(CSolution.ucur_x)
                u_pre_y = deepcopy(CSolution.ucur_y)

                Momentum.runge_kutta(CSolution, RK_MAX_ITERATION, RK_TOLERANCE, CONVECTIVE_COEFFICIENT)
                phi = Continuity.poisson_solv(CSolution)
                Updater.solv_update(CSolution, phi)

                u_new_x = deepcopy(CSolution.ucur_x)
                u_new_y = deepcopy(CSolution.ucur_y)

                global CSolution.du_x = u_new_x - u_pre_x
                global CSolution.du_y = u_new_y - u_pre_y
            end

            FormBCS.form_bcs(CSolution)
            global timestep += 1
        \end{minted}
    \end{center}
    
\end{frame}
%   FRAME END   --==-=-=-=-=-=-=-=-=-=-=-=-=-=-=-=-=-=-=-=-=-=-=-=-=-=-=-=-=-=-=
\end{document}